\documentclass[11pt]{article}

    \usepackage[breakable]{tcolorbox}
    \usepackage{parskip} % Stop auto-indenting (to mimic markdown behaviour)
    

    % Basic figure setup, for now with no caption control since it's done
    % automatically by Pandoc (which extracts ![](path) syntax from Markdown).
    \usepackage{graphicx}
    % Maintain compatibility with old templates. Remove in nbconvert 6.0
    \let\Oldincludegraphics\includegraphics
    % Ensure that by default, figures have no caption (until we provide a
    % proper Figure object with a Caption API and a way to capture that
    % in the conversion process - todo).
    \usepackage{caption}
    \DeclareCaptionFormat{nocaption}{}
    \captionsetup{format=nocaption,aboveskip=0pt,belowskip=0pt}

    \usepackage{float}
    \floatplacement{figure}{H} % forces figures to be placed at the correct location
    \usepackage{xcolor} % Allow colors to be defined
    \usepackage{enumerate} % Needed for markdown enumerations to work
    \usepackage{geometry} % Used to adjust the document margins
    \usepackage{amsmath} % Equations
    \usepackage{amssymb} % Equations
    \usepackage{textcomp} % defines textquotesingle
    % Hack from http://tex.stackexchange.com/a/47451/13684:
    \AtBeginDocument{%
        \def\PYZsq{\textquotesingle}% Upright quotes in Pygmentized code
    }
    \usepackage{upquote} % Upright quotes for verbatim code
    \usepackage{eurosym} % defines \euro

    \usepackage{iftex}
    \ifPDFTeX
        \usepackage[T1]{fontenc}
        \IfFileExists{alphabeta.sty}{
              \usepackage{alphabeta}
          }{
              \usepackage[mathletters]{ucs}
              \usepackage[utf8x]{inputenc}
          }
    \else
        \usepackage{fontspec}
        \usepackage{unicode-math}
    \fi

    \usepackage{fancyvrb} % verbatim replacement that allows latex
    \usepackage{grffile} % extends the file name processing of package graphics
                         % to support a larger range
    \makeatletter % fix for old versions of grffile with XeLaTeX
    \@ifpackagelater{grffile}{2019/11/01}
    {
      % Do nothing on new versions
    }
    {
      \def\Gread@@xetex#1{%
        \IfFileExists{"\Gin@base".bb}%
        {\Gread@eps{\Gin@base.bb}}%
        {\Gread@@xetex@aux#1}%
      }
    }
    \makeatother
    \usepackage[Export]{adjustbox} % Used to constrain images to a maximum size
    \adjustboxset{max size={0.9\linewidth}{0.9\paperheight}}

    % The hyperref package gives us a pdf with properly built
    % internal navigation ('pdf bookmarks' for the table of contents,
    % internal cross-reference links, web links for URLs, etc.)
    \usepackage{hyperref}
    % The default LaTeX title has an obnoxious amount of whitespace. By default,
    % titling removes some of it. It also provides customization options.
    \usepackage{titling}
    \usepackage{longtable} % longtable support required by pandoc >1.10
    \usepackage{booktabs}  % table support for pandoc > 1.12.2
    \usepackage{array}     % table support for pandoc >= 2.11.3
    \usepackage{calc}      % table minipage width calculation for pandoc >= 2.11.1
    \usepackage[inline]{enumitem} % IRkernel/repr support (it uses the enumerate* environment)
    \usepackage[normalem]{ulem} % ulem is needed to support strikethroughs (\sout)
                                % normalem makes italics be italics, not underlines
    \usepackage{mathrsfs}
    

    
    % Colors for the hyperref package
    \definecolor{urlcolor}{rgb}{0,.145,.698}
    \definecolor{linkcolor}{rgb}{.71,0.21,0.01}
    \definecolor{citecolor}{rgb}{.12,.54,.11}

    % ANSI colors
    \definecolor{ansi-black}{HTML}{3E424D}
    \definecolor{ansi-black-intense}{HTML}{282C36}
    \definecolor{ansi-red}{HTML}{E75C58}
    \definecolor{ansi-red-intense}{HTML}{B22B31}
    \definecolor{ansi-green}{HTML}{00A250}
    \definecolor{ansi-green-intense}{HTML}{007427}
    \definecolor{ansi-yellow}{HTML}{DDB62B}
    \definecolor{ansi-yellow-intense}{HTML}{B27D12}
    \definecolor{ansi-blue}{HTML}{208FFB}
    \definecolor{ansi-blue-intense}{HTML}{0065CA}
    \definecolor{ansi-magenta}{HTML}{D160C4}
    \definecolor{ansi-magenta-intense}{HTML}{A03196}
    \definecolor{ansi-cyan}{HTML}{60C6C8}
    \definecolor{ansi-cyan-intense}{HTML}{258F8F}
    \definecolor{ansi-white}{HTML}{C5C1B4}
    \definecolor{ansi-white-intense}{HTML}{A1A6B2}
    \definecolor{ansi-default-inverse-fg}{HTML}{FFFFFF}
    \definecolor{ansi-default-inverse-bg}{HTML}{000000}

    % common color for the border for error outputs.
    \definecolor{outerrorbackground}{HTML}{FFDFDF}

    % commands and environments needed by pandoc snippets
    % extracted from the output of `pandoc -s`
    \providecommand{\tightlist}{%
      \setlength{\itemsep}{0pt}\setlength{\parskip}{0pt}}
    \DefineVerbatimEnvironment{Highlighting}{Verbatim}{commandchars=\\\{\}}
    % Add ',fontsize=\small' for more characters per line
    \newenvironment{Shaded}{}{}
    \newcommand{\KeywordTok}[1]{\textcolor[rgb]{0.00,0.44,0.13}{\textbf{{#1}}}}
    \newcommand{\DataTypeTok}[1]{\textcolor[rgb]{0.56,0.13,0.00}{{#1}}}
    \newcommand{\DecValTok}[1]{\textcolor[rgb]{0.25,0.63,0.44}{{#1}}}
    \newcommand{\BaseNTok}[1]{\textcolor[rgb]{0.25,0.63,0.44}{{#1}}}
    \newcommand{\FloatTok}[1]{\textcolor[rgb]{0.25,0.63,0.44}{{#1}}}
    \newcommand{\CharTok}[1]{\textcolor[rgb]{0.25,0.44,0.63}{{#1}}}
    \newcommand{\StringTok}[1]{\textcolor[rgb]{0.25,0.44,0.63}{{#1}}}
    \newcommand{\CommentTok}[1]{\textcolor[rgb]{0.38,0.63,0.69}{\textit{{#1}}}}
    \newcommand{\OtherTok}[1]{\textcolor[rgb]{0.00,0.44,0.13}{{#1}}}
    \newcommand{\AlertTok}[1]{\textcolor[rgb]{1.00,0.00,0.00}{\textbf{{#1}}}}
    \newcommand{\FunctionTok}[1]{\textcolor[rgb]{0.02,0.16,0.49}{{#1}}}
    \newcommand{\RegionMarkerTok}[1]{{#1}}
    \newcommand{\ErrorTok}[1]{\textcolor[rgb]{1.00,0.00,0.00}{\textbf{{#1}}}}
    \newcommand{\NormalTok}[1]{{#1}}

    % Additional commands for more recent versions of Pandoc
    \newcommand{\ConstantTok}[1]{\textcolor[rgb]{0.53,0.00,0.00}{{#1}}}
    \newcommand{\SpecialCharTok}[1]{\textcolor[rgb]{0.25,0.44,0.63}{{#1}}}
    \newcommand{\VerbatimStringTok}[1]{\textcolor[rgb]{0.25,0.44,0.63}{{#1}}}
    \newcommand{\SpecialStringTok}[1]{\textcolor[rgb]{0.73,0.40,0.53}{{#1}}}
    \newcommand{\ImportTok}[1]{{#1}}
    \newcommand{\DocumentationTok}[1]{\textcolor[rgb]{0.73,0.13,0.13}{\textit{{#1}}}}
    \newcommand{\AnnotationTok}[1]{\textcolor[rgb]{0.38,0.63,0.69}{\textbf{\textit{{#1}}}}}
    \newcommand{\CommentVarTok}[1]{\textcolor[rgb]{0.38,0.63,0.69}{\textbf{\textit{{#1}}}}}
    \newcommand{\VariableTok}[1]{\textcolor[rgb]{0.10,0.09,0.49}{{#1}}}
    \newcommand{\ControlFlowTok}[1]{\textcolor[rgb]{0.00,0.44,0.13}{\textbf{{#1}}}}
    \newcommand{\OperatorTok}[1]{\textcolor[rgb]{0.40,0.40,0.40}{{#1}}}
    \newcommand{\BuiltInTok}[1]{{#1}}
    \newcommand{\ExtensionTok}[1]{{#1}}
    \newcommand{\PreprocessorTok}[1]{\textcolor[rgb]{0.74,0.48,0.00}{{#1}}}
    \newcommand{\AttributeTok}[1]{\textcolor[rgb]{0.49,0.56,0.16}{{#1}}}
    \newcommand{\InformationTok}[1]{\textcolor[rgb]{0.38,0.63,0.69}{\textbf{\textit{{#1}}}}}
    \newcommand{\WarningTok}[1]{\textcolor[rgb]{0.38,0.63,0.69}{\textbf{\textit{{#1}}}}}


    % Define a nice break command that doesn't care if a line doesn't already
    % exist.
    \def\br{\hspace*{\fill} \\* }
    % Math Jax compatibility definitions
    \def\gt{>}
    \def\lt{<}
    \let\Oldtex\TeX
    \let\Oldlatex\LaTeX
    \renewcommand{\TeX}{\textrm{\Oldtex}}
    \renewcommand{\LaTeX}{\textrm{\Oldlatex}}
    % Document parameters
    % Document title
    \title{Assignment 3}
    
    
    
    
    
% Pygments definitions
\makeatletter
\def\PY@reset{\let\PY@it=\relax \let\PY@bf=\relax%
    \let\PY@ul=\relax \let\PY@tc=\relax%
    \let\PY@bc=\relax \let\PY@ff=\relax}
\def\PY@tok#1{\csname PY@tok@#1\endcsname}
\def\PY@toks#1+{\ifx\relax#1\empty\else%
    \PY@tok{#1}\expandafter\PY@toks\fi}
\def\PY@do#1{\PY@bc{\PY@tc{\PY@ul{%
    \PY@it{\PY@bf{\PY@ff{#1}}}}}}}
\def\PY#1#2{\PY@reset\PY@toks#1+\relax+\PY@do{#2}}

\@namedef{PY@tok@w}{\def\PY@tc##1{\textcolor[rgb]{0.73,0.73,0.73}{##1}}}
\@namedef{PY@tok@c}{\let\PY@it=\textit\def\PY@tc##1{\textcolor[rgb]{0.24,0.48,0.48}{##1}}}
\@namedef{PY@tok@cp}{\def\PY@tc##1{\textcolor[rgb]{0.61,0.40,0.00}{##1}}}
\@namedef{PY@tok@k}{\let\PY@bf=\textbf\def\PY@tc##1{\textcolor[rgb]{0.00,0.50,0.00}{##1}}}
\@namedef{PY@tok@kp}{\def\PY@tc##1{\textcolor[rgb]{0.00,0.50,0.00}{##1}}}
\@namedef{PY@tok@kt}{\def\PY@tc##1{\textcolor[rgb]{0.69,0.00,0.25}{##1}}}
\@namedef{PY@tok@o}{\def\PY@tc##1{\textcolor[rgb]{0.40,0.40,0.40}{##1}}}
\@namedef{PY@tok@ow}{\let\PY@bf=\textbf\def\PY@tc##1{\textcolor[rgb]{0.67,0.13,1.00}{##1}}}
\@namedef{PY@tok@nb}{\def\PY@tc##1{\textcolor[rgb]{0.00,0.50,0.00}{##1}}}
\@namedef{PY@tok@nf}{\def\PY@tc##1{\textcolor[rgb]{0.00,0.00,1.00}{##1}}}
\@namedef{PY@tok@nc}{\let\PY@bf=\textbf\def\PY@tc##1{\textcolor[rgb]{0.00,0.00,1.00}{##1}}}
\@namedef{PY@tok@nn}{\let\PY@bf=\textbf\def\PY@tc##1{\textcolor[rgb]{0.00,0.00,1.00}{##1}}}
\@namedef{PY@tok@ne}{\let\PY@bf=\textbf\def\PY@tc##1{\textcolor[rgb]{0.80,0.25,0.22}{##1}}}
\@namedef{PY@tok@nv}{\def\PY@tc##1{\textcolor[rgb]{0.10,0.09,0.49}{##1}}}
\@namedef{PY@tok@no}{\def\PY@tc##1{\textcolor[rgb]{0.53,0.00,0.00}{##1}}}
\@namedef{PY@tok@nl}{\def\PY@tc##1{\textcolor[rgb]{0.46,0.46,0.00}{##1}}}
\@namedef{PY@tok@ni}{\let\PY@bf=\textbf\def\PY@tc##1{\textcolor[rgb]{0.44,0.44,0.44}{##1}}}
\@namedef{PY@tok@na}{\def\PY@tc##1{\textcolor[rgb]{0.41,0.47,0.13}{##1}}}
\@namedef{PY@tok@nt}{\let\PY@bf=\textbf\def\PY@tc##1{\textcolor[rgb]{0.00,0.50,0.00}{##1}}}
\@namedef{PY@tok@nd}{\def\PY@tc##1{\textcolor[rgb]{0.67,0.13,1.00}{##1}}}
\@namedef{PY@tok@s}{\def\PY@tc##1{\textcolor[rgb]{0.73,0.13,0.13}{##1}}}
\@namedef{PY@tok@sd}{\let\PY@it=\textit\def\PY@tc##1{\textcolor[rgb]{0.73,0.13,0.13}{##1}}}
\@namedef{PY@tok@si}{\let\PY@bf=\textbf\def\PY@tc##1{\textcolor[rgb]{0.64,0.35,0.47}{##1}}}
\@namedef{PY@tok@se}{\let\PY@bf=\textbf\def\PY@tc##1{\textcolor[rgb]{0.67,0.36,0.12}{##1}}}
\@namedef{PY@tok@sr}{\def\PY@tc##1{\textcolor[rgb]{0.64,0.35,0.47}{##1}}}
\@namedef{PY@tok@ss}{\def\PY@tc##1{\textcolor[rgb]{0.10,0.09,0.49}{##1}}}
\@namedef{PY@tok@sx}{\def\PY@tc##1{\textcolor[rgb]{0.00,0.50,0.00}{##1}}}
\@namedef{PY@tok@m}{\def\PY@tc##1{\textcolor[rgb]{0.40,0.40,0.40}{##1}}}
\@namedef{PY@tok@gh}{\let\PY@bf=\textbf\def\PY@tc##1{\textcolor[rgb]{0.00,0.00,0.50}{##1}}}
\@namedef{PY@tok@gu}{\let\PY@bf=\textbf\def\PY@tc##1{\textcolor[rgb]{0.50,0.00,0.50}{##1}}}
\@namedef{PY@tok@gd}{\def\PY@tc##1{\textcolor[rgb]{0.63,0.00,0.00}{##1}}}
\@namedef{PY@tok@gi}{\def\PY@tc##1{\textcolor[rgb]{0.00,0.52,0.00}{##1}}}
\@namedef{PY@tok@gr}{\def\PY@tc##1{\textcolor[rgb]{0.89,0.00,0.00}{##1}}}
\@namedef{PY@tok@ge}{\let\PY@it=\textit}
\@namedef{PY@tok@gs}{\let\PY@bf=\textbf}
\@namedef{PY@tok@gp}{\let\PY@bf=\textbf\def\PY@tc##1{\textcolor[rgb]{0.00,0.00,0.50}{##1}}}
\@namedef{PY@tok@go}{\def\PY@tc##1{\textcolor[rgb]{0.44,0.44,0.44}{##1}}}
\@namedef{PY@tok@gt}{\def\PY@tc##1{\textcolor[rgb]{0.00,0.27,0.87}{##1}}}
\@namedef{PY@tok@err}{\def\PY@bc##1{{\setlength{\fboxsep}{\string -\fboxrule}\fcolorbox[rgb]{1.00,0.00,0.00}{1,1,1}{\strut ##1}}}}
\@namedef{PY@tok@kc}{\let\PY@bf=\textbf\def\PY@tc##1{\textcolor[rgb]{0.00,0.50,0.00}{##1}}}
\@namedef{PY@tok@kd}{\let\PY@bf=\textbf\def\PY@tc##1{\textcolor[rgb]{0.00,0.50,0.00}{##1}}}
\@namedef{PY@tok@kn}{\let\PY@bf=\textbf\def\PY@tc##1{\textcolor[rgb]{0.00,0.50,0.00}{##1}}}
\@namedef{PY@tok@kr}{\let\PY@bf=\textbf\def\PY@tc##1{\textcolor[rgb]{0.00,0.50,0.00}{##1}}}
\@namedef{PY@tok@bp}{\def\PY@tc##1{\textcolor[rgb]{0.00,0.50,0.00}{##1}}}
\@namedef{PY@tok@fm}{\def\PY@tc##1{\textcolor[rgb]{0.00,0.00,1.00}{##1}}}
\@namedef{PY@tok@vc}{\def\PY@tc##1{\textcolor[rgb]{0.10,0.09,0.49}{##1}}}
\@namedef{PY@tok@vg}{\def\PY@tc##1{\textcolor[rgb]{0.10,0.09,0.49}{##1}}}
\@namedef{PY@tok@vi}{\def\PY@tc##1{\textcolor[rgb]{0.10,0.09,0.49}{##1}}}
\@namedef{PY@tok@vm}{\def\PY@tc##1{\textcolor[rgb]{0.10,0.09,0.49}{##1}}}
\@namedef{PY@tok@sa}{\def\PY@tc##1{\textcolor[rgb]{0.73,0.13,0.13}{##1}}}
\@namedef{PY@tok@sb}{\def\PY@tc##1{\textcolor[rgb]{0.73,0.13,0.13}{##1}}}
\@namedef{PY@tok@sc}{\def\PY@tc##1{\textcolor[rgb]{0.73,0.13,0.13}{##1}}}
\@namedef{PY@tok@dl}{\def\PY@tc##1{\textcolor[rgb]{0.73,0.13,0.13}{##1}}}
\@namedef{PY@tok@s2}{\def\PY@tc##1{\textcolor[rgb]{0.73,0.13,0.13}{##1}}}
\@namedef{PY@tok@sh}{\def\PY@tc##1{\textcolor[rgb]{0.73,0.13,0.13}{##1}}}
\@namedef{PY@tok@s1}{\def\PY@tc##1{\textcolor[rgb]{0.73,0.13,0.13}{##1}}}
\@namedef{PY@tok@mb}{\def\PY@tc##1{\textcolor[rgb]{0.40,0.40,0.40}{##1}}}
\@namedef{PY@tok@mf}{\def\PY@tc##1{\textcolor[rgb]{0.40,0.40,0.40}{##1}}}
\@namedef{PY@tok@mh}{\def\PY@tc##1{\textcolor[rgb]{0.40,0.40,0.40}{##1}}}
\@namedef{PY@tok@mi}{\def\PY@tc##1{\textcolor[rgb]{0.40,0.40,0.40}{##1}}}
\@namedef{PY@tok@il}{\def\PY@tc##1{\textcolor[rgb]{0.40,0.40,0.40}{##1}}}
\@namedef{PY@tok@mo}{\def\PY@tc##1{\textcolor[rgb]{0.40,0.40,0.40}{##1}}}
\@namedef{PY@tok@ch}{\let\PY@it=\textit\def\PY@tc##1{\textcolor[rgb]{0.24,0.48,0.48}{##1}}}
\@namedef{PY@tok@cm}{\let\PY@it=\textit\def\PY@tc##1{\textcolor[rgb]{0.24,0.48,0.48}{##1}}}
\@namedef{PY@tok@cpf}{\let\PY@it=\textit\def\PY@tc##1{\textcolor[rgb]{0.24,0.48,0.48}{##1}}}
\@namedef{PY@tok@c1}{\let\PY@it=\textit\def\PY@tc##1{\textcolor[rgb]{0.24,0.48,0.48}{##1}}}
\@namedef{PY@tok@cs}{\let\PY@it=\textit\def\PY@tc##1{\textcolor[rgb]{0.24,0.48,0.48}{##1}}}

\def\PYZbs{\char`\\}
\def\PYZus{\char`\_}
\def\PYZob{\char`\{}
\def\PYZcb{\char`\}}
\def\PYZca{\char`\^}
\def\PYZam{\char`\&}
\def\PYZlt{\char`\<}
\def\PYZgt{\char`\>}
\def\PYZsh{\char`\#}
\def\PYZpc{\char`\%}
\def\PYZdl{\char`\$}
\def\PYZhy{\char`\-}
\def\PYZsq{\char`\'}
\def\PYZdq{\char`\"}
\def\PYZti{\char`\~}
% for compatibility with earlier versions
\def\PYZat{@}
\def\PYZlb{[}
\def\PYZrb{]}
\makeatother


    % For linebreaks inside Verbatim environment from package fancyvrb.
    \makeatletter
        \newbox\Wrappedcontinuationbox
        \newbox\Wrappedvisiblespacebox
        \newcommand*\Wrappedvisiblespace {\textcolor{red}{\textvisiblespace}}
        \newcommand*\Wrappedcontinuationsymbol {\textcolor{red}{\llap{\tiny$\m@th\hookrightarrow$}}}
        \newcommand*\Wrappedcontinuationindent {3ex }
        \newcommand*\Wrappedafterbreak {\kern\Wrappedcontinuationindent\copy\Wrappedcontinuationbox}
        % Take advantage of the already applied Pygments mark-up to insert
        % potential linebreaks for TeX processing.
        %        {, <, #, %, $, ' and ": go to next line.
        %        _, }, ^, &, >, - and ~: stay at end of broken line.
        % Use of \textquotesingle for straight quote.
        \newcommand*\Wrappedbreaksatspecials {%
            \def\PYGZus{\discretionary{\char`\_}{\Wrappedafterbreak}{\char`\_}}%
            \def\PYGZob{\discretionary{}{\Wrappedafterbreak\char`\{}{\char`\{}}%
            \def\PYGZcb{\discretionary{\char`\}}{\Wrappedafterbreak}{\char`\}}}%
            \def\PYGZca{\discretionary{\char`\^}{\Wrappedafterbreak}{\char`\^}}%
            \def\PYGZam{\discretionary{\char`\&}{\Wrappedafterbreak}{\char`\&}}%
            \def\PYGZlt{\discretionary{}{\Wrappedafterbreak\char`\<}{\char`\<}}%
            \def\PYGZgt{\discretionary{\char`\>}{\Wrappedafterbreak}{\char`\>}}%
            \def\PYGZsh{\discretionary{}{\Wrappedafterbreak\char`\#}{\char`\#}}%
            \def\PYGZpc{\discretionary{}{\Wrappedafterbreak\char`\%}{\char`\%}}%
            \def\PYGZdl{\discretionary{}{\Wrappedafterbreak\char`\$}{\char`\$}}%
            \def\PYGZhy{\discretionary{\char`\-}{\Wrappedafterbreak}{\char`\-}}%
            \def\PYGZsq{\discretionary{}{\Wrappedafterbreak\textquotesingle}{\textquotesingle}}%
            \def\PYGZdq{\discretionary{}{\Wrappedafterbreak\char`\"}{\char`\"}}%
            \def\PYGZti{\discretionary{\char`\~}{\Wrappedafterbreak}{\char`\~}}%
        }
        % Some characters . , ; ? ! / are not pygmentized.
        % This macro makes them "active" and they will insert potential linebreaks
        \newcommand*\Wrappedbreaksatpunct {%
            \lccode`\~`\.\lowercase{\def~}{\discretionary{\hbox{\char`\.}}{\Wrappedafterbreak}{\hbox{\char`\.}}}%
            \lccode`\~`\,\lowercase{\def~}{\discretionary{\hbox{\char`\,}}{\Wrappedafterbreak}{\hbox{\char`\,}}}%
            \lccode`\~`\;\lowercase{\def~}{\discretionary{\hbox{\char`\;}}{\Wrappedafterbreak}{\hbox{\char`\;}}}%
            \lccode`\~`\:\lowercase{\def~}{\discretionary{\hbox{\char`\:}}{\Wrappedafterbreak}{\hbox{\char`\:}}}%
            \lccode`\~`\?\lowercase{\def~}{\discretionary{\hbox{\char`\?}}{\Wrappedafterbreak}{\hbox{\char`\?}}}%
            \lccode`\~`\!\lowercase{\def~}{\discretionary{\hbox{\char`\!}}{\Wrappedafterbreak}{\hbox{\char`\!}}}%
            \lccode`\~`\/\lowercase{\def~}{\discretionary{\hbox{\char`\/}}{\Wrappedafterbreak}{\hbox{\char`\/}}}%
            \catcode`\.\active
            \catcode`\,\active
            \catcode`\;\active
            \catcode`\:\active
            \catcode`\?\active
            \catcode`\!\active
            \catcode`\/\active
            \lccode`\~`\~
        }
    \makeatother

    \let\OriginalVerbatim=\Verbatim
    \makeatletter
    \renewcommand{\Verbatim}[1][1]{%
        %\parskip\z@skip
        \sbox\Wrappedcontinuationbox {\Wrappedcontinuationsymbol}%
        \sbox\Wrappedvisiblespacebox {\FV@SetupFont\Wrappedvisiblespace}%
        \def\FancyVerbFormatLine ##1{\hsize\linewidth
            \vtop{\raggedright\hyphenpenalty\z@\exhyphenpenalty\z@
                \doublehyphendemerits\z@\finalhyphendemerits\z@
                \strut ##1\strut}%
        }%
        % If the linebreak is at a space, the latter will be displayed as visible
        % space at end of first line, and a continuation symbol starts next line.
        % Stretch/shrink are however usually zero for typewriter font.
        \def\FV@Space {%
            \nobreak\hskip\z@ plus\fontdimen3\font minus\fontdimen4\font
            \discretionary{\copy\Wrappedvisiblespacebox}{\Wrappedafterbreak}
            {\kern\fontdimen2\font}%
        }%

        % Allow breaks at special characters using \PYG... macros.
        \Wrappedbreaksatspecials
        % Breaks at punctuation characters . , ; ? ! and / need catcode=\active
        \OriginalVerbatim[#1,codes*=\Wrappedbreaksatpunct]%
    }
    \makeatother

    % Exact colors from NB
    \definecolor{incolor}{HTML}{303F9F}
    \definecolor{outcolor}{HTML}{D84315}
    \definecolor{cellborder}{HTML}{CFCFCF}
    \definecolor{cellbackground}{HTML}{F7F7F7}

    % prompt
    \makeatletter
    \newcommand{\boxspacing}{\kern\kvtcb@left@rule\kern\kvtcb@boxsep}
    \makeatother
    \newcommand{\prompt}[4]{
        {\ttfamily\llap{{\color{#2}[#3]:\hspace{3pt}#4}}\vspace{-\baselineskip}}
    }
    

    
    % Prevent overflowing lines due to hard-to-break entities
    \sloppy
    % Setup hyperref package
    \hypersetup{
      breaklinks=true,  % so long urls are correctly broken across lines
      colorlinks=true,
      urlcolor=urlcolor,
      linkcolor=linkcolor,
      citecolor=citecolor,
      }
    % Slightly bigger margins than the latex defaults
    
    \geometry{verbose,tmargin=1in,bmargin=1in,lmargin=1in,rmargin=1in}
    
    

\begin{document}
    
    \maketitle
    
    

    
    \section{Assignment 3}\label{assignment-3}

\subsubsection{Filip Sjövall 19950308}\label{filip-sjuxf6vall-19950308}

    \subsection{Task 1}\label{task-1}

The task is to find a polynomial representation of some sample data
using the least squares method. The residual \(r = Ax-b\) will be
minimized if it is orthogonal to the column space of \(A\) /
range(\(A\)),m i.e. \(A^T r = 0\). This gives us the normal form
\(A^T A x = A^T b\) for which we will seek a solution in this task.

The input data is modelled by

\begin{equation}
y(x) = a sin(x) + b cos(x) + c sin(2x) + dcos(2x)
\end{equation}

where \(a,b,c,d\) are the unknowns. Given the sample data \((x_i,y_i)\)
the following system of equations models the solution

\begin{equation}
Ax = y 
\label{eq:lsq_sys}
,
\end{equation}

where \$ x = (a,b,c,d)\^{}T \$, y the sampled data \(y_i\) and

\begin{equation}
A =
\begin{bmatrix}
sin(x_1) & cos(x_1) & sin(2x_1) & cos(2x_1) \\
sin(x_2) & cos(x_2) & sin(2x_2) & cos(2x_2) \\
. & . & . & . \\
. & . & . & . \\
. & . & . & . \\
sin(x_n) & cos(x_n) & sin(2x_n) & cos(2x_n) \\
\end{bmatrix}.
\end{equation}

The idea is to solve the system (\ref{eq:lsq_sys}) not using \(A^{-1}\)
which is expensive but instead using QR or SVD. To get square matrices
we can multiply from the left to get

\begin{equation}\label{eq:}
A^T A x = A^T y = b,
\end{equation}

so then \(QR(A^T A)\) gives us

\begin{align}
QR x &= b, \\
R  x &= Q^T b, 
\end{align}

which can be solved using backwards substitution. Similarly for the
SVD(A\^{}T A) we get

\begin{align}
U\Sigma V^T x &= b , \\
            x &= V \Sigma ^ {-1} U^T b
\end{align}

which is also straight forward to solve (\(\Sigma\) is diagonal so the
inverse is diag(1/\sigma\_i...)) and the remaining matrix vector
products are also straight forward. The cost of solving
\(\ref{eq:lsq_sys}\) using the above described methods are in both cases
dominated by the factorization as the rest is matrix-vector
multiplications or backwards substitution. Något om hur r vinkelrät mot
range(A) och A\^{}Tr=0

    \begin{tcolorbox}[breakable, size=fbox, boxrule=1pt, pad at break*=1mm,colback=cellbackground, colframe=cellborder]
\prompt{In}{incolor}{214}{\boxspacing}
\begin{Verbatim}[commandchars=\\\{\}]
\PY{k+kn}{from} \PY{n+nn}{numpy} \PY{k+kn}{import} \PY{o}{*}
\PY{c+c1}{\PYZsh{}from scipy import *}
\PY{k+kn}{from} \PY{n+nn}{numpy} \PY{k+kn}{import} \PY{n}{linalg}
\PY{c+c1}{\PYZsh{}from scipy import linalg}
\PY{k+kn}{from} \PY{n+nn}{scipy} \PY{k+kn}{import} \PY{n}{optimize}
\PY{k+kn}{import} \PY{n+nn}{matplotlib}\PY{n+nn}{.}\PY{n+nn}{pyplot} \PY{k}{as} \PY{n+nn}{plt}
\PY{k+kn}{from} \PY{n+nn}{matplotlib}\PY{n+nn}{.}\PY{n+nn}{ticker} \PY{k+kn}{import} \PY{n}{FuncFormatter}\PY{p}{,} \PY{n}{MultipleLocator}
\PY{k+kn}{import} \PY{n+nn}{math}
\PY{k+kn}{import} \PY{n+nn}{sys}\PY{o}{,} \PY{n+nn}{getopt}\PY{o}{,} \PY{n+nn}{os}\PY{o}{,} \PY{n+nn}{string}
\end{Verbatim}
\end{tcolorbox}

    \begin{tcolorbox}[breakable, size=fbox, boxrule=1pt, pad at break*=1mm,colback=cellbackground, colframe=cellborder]
\prompt{In}{incolor}{215}{\boxspacing}
\begin{Verbatim}[commandchars=\\\{\}]
\PY{c+c1}{\PYZsh{} Open file and create vectors x and y}
\PY{n}{file} \PY{o}{=} \PY{n+nb}{open}\PY{p}{(}\PY{l+s+s1}{\PYZsq{}}\PY{l+s+s1}{/home/filip/Documents/NumLinAlg/Assignment 3/signal.dat}\PY{l+s+s1}{\PYZsq{}}\PY{p}{,}\PY{l+s+s1}{\PYZsq{}}\PY{l+s+s1}{r}\PY{l+s+s1}{\PYZsq{}}\PY{p}{)}
\PY{n}{x0} \PY{o}{=} \PY{p}{[}\PY{p}{]}
\PY{n}{y} \PY{o}{=} \PY{p}{[}\PY{p}{]}
\PY{k}{for} \PY{n}{row}\PY{p}{,} \PY{n}{line} \PY{o+ow}{in} \PY{n+nb}{enumerate}\PY{p}{(}\PY{n}{file}\PY{p}{,} \PY{l+m+mi}{1}\PY{p}{)}\PY{p}{:}
    \PY{n}{fname} \PY{o}{=} \PY{n}{line}\PY{o}{.}\PY{n}{rstrip}\PY{p}{(}\PY{p}{)}\PY{o}{.}\PY{n}{split}\PY{p}{(}\PY{l+s+s1}{\PYZsq{}}\PY{l+s+s1}{,}\PY{l+s+s1}{\PYZsq{}}\PY{p}{)} \PY{c+c1}{\PYZsh{}using rstrip to remove the \PYZbs{}n}
    \PY{n}{x0}\PY{o}{.}\PY{n}{append}\PY{p}{(}\PY{n}{float64}\PY{p}{(}\PY{n}{fname}\PY{p}{[}\PY{l+m+mi}{0}\PY{p}{]}\PY{p}{)}\PY{p}{)}
    \PY{n}{y}\PY{o}{.}\PY{n}{append}\PY{p}{(}\PY{n}{float64}\PY{p}{(}\PY{n}{fname}\PY{p}{[}\PY{l+m+mi}{1}\PY{p}{]}\PY{p}{)}\PY{p}{)}
\end{Verbatim}
\end{tcolorbox}

    \begin{tcolorbox}[breakable, size=fbox, boxrule=1pt, pad at break*=1mm,colback=cellbackground, colframe=cellborder]
\prompt{In}{incolor}{244}{\boxspacing}
\begin{Verbatim}[commandchars=\\\{\}]
\PY{c+c1}{\PYZsh{} Create A for the linear system}
\PY{n}{A0} \PY{o}{=} \PY{n}{zeros}\PY{p}{(}\PY{p}{[}\PY{n+nb}{len}\PY{p}{(}\PY{n}{x0}\PY{p}{)}\PY{p}{,}\PY{l+m+mi}{4}\PY{p}{]}\PY{p}{)}
\PY{k}{for} \PY{n}{i}\PY{p}{,} \PY{n}{xi} \PY{o+ow}{in} \PY{n+nb}{enumerate}\PY{p}{(}\PY{n}{x0}\PY{p}{,} \PY{l+m+mi}{1} \PY{p}{)}\PY{p}{:}
    \PY{n}{A0}\PY{p}{[}\PY{n+nb}{int}\PY{p}{(}\PY{n}{i}\PY{p}{)}\PY{o}{\PYZhy{}}\PY{l+m+mi}{1}\PY{p}{,}\PY{p}{:}\PY{p}{]} \PY{o}{=} \PY{p}{[}\PY{n}{sin}\PY{p}{(}\PY{n}{xi}\PY{p}{)}\PY{p}{,}\PY{n}{cos}\PY{p}{(}\PY{n}{xi}\PY{p}{)}\PY{p}{,}\PY{n}{sin}\PY{p}{(}\PY{l+m+mi}{2}\PY{o}{*}\PY{n}{xi}\PY{p}{)}\PY{p}{,}\PY{n}{cos}\PY{p}{(}\PY{l+m+mi}{2}\PY{o}{*}\PY{n}{xi}\PY{p}{)}\PY{p}{]}
\end{Verbatim}
\end{tcolorbox}

    The system is solved both with QR factorization and with Singular value
decomposition

    \begin{tcolorbox}[breakable, size=fbox, boxrule=1pt, pad at break*=1mm,colback=cellbackground, colframe=cellborder]
\prompt{In}{incolor}{218}{\boxspacing}
\begin{Verbatim}[commandchars=\\\{\}]
\PY{n+nb}{print}\PY{p}{(}\PY{n}{shape}\PY{p}{(}\PY{n}{A0}\PY{p}{)}\PY{p}{)}
\PY{n}{A} \PY{o}{=} \PY{n}{A0}\PY{o}{.}\PY{n}{T}\PY{n+nd}{@A0}
\PY{n}{b0} \PY{o}{=} \PY{n}{A0}\PY{o}{.}\PY{n}{T}\PY{n+nd}{@y}
\PY{n}{Q}\PY{p}{,}\PY{n}{R} \PY{o}{=} \PY{n}{linalg}\PY{o}{.}\PY{n}{qr}\PY{p}{(}\PY{n}{A}\PY{p}{)}

\PY{n}{b} \PY{o}{=} \PY{n}{Q}\PY{n+nd}{@b0}

\PY{c+c1}{\PYZsh{}System solution, could probably find a closed form solution to loop over for models with more parameters in x}
\PY{n}{x}     \PY{o}{=} \PY{n}{zeros}\PY{p}{(}\PY{p}{[}\PY{l+m+mi}{4}\PY{p}{,}\PY{l+m+mi}{1}\PY{p}{]}\PY{p}{)}
\PY{n}{x}\PY{p}{[}\PY{l+m+mi}{3}\PY{p}{]} \PY{o}{=}   \PY{n}{b}\PY{p}{[}\PY{l+m+mi}{3}\PY{p}{]}\PY{o}{/}\PY{n}{R}\PY{p}{[}\PY{l+m+mi}{3}\PY{p}{,}\PY{l+m+mi}{3}\PY{p}{]}
\PY{n}{x}\PY{p}{[}\PY{l+m+mi}{2}\PY{p}{]}  \PY{o}{=} \PY{p}{(}\PY{n}{b}\PY{p}{[}\PY{l+m+mi}{2}\PY{p}{]} \PY{o}{\PYZhy{}} \PY{n}{R}\PY{p}{[}\PY{l+m+mi}{2}\PY{p}{,}\PY{l+m+mi}{3}\PY{p}{]}\PY{o}{*}\PY{n}{x}\PY{p}{[}\PY{l+m+mi}{3}\PY{p}{]} \PY{p}{)}\PY{o}{/}\PY{n}{R}\PY{p}{[}\PY{l+m+mi}{2}\PY{p}{,}\PY{l+m+mi}{2}\PY{p}{]}
\PY{n}{x}\PY{p}{[}\PY{l+m+mi}{1}\PY{p}{]}  \PY{o}{=} \PY{p}{(}\PY{n}{b}\PY{p}{[}\PY{l+m+mi}{1}\PY{p}{]} \PY{o}{\PYZhy{}} \PY{n}{R}\PY{p}{[}\PY{l+m+mi}{1}\PY{p}{,}\PY{l+m+mi}{2}\PY{p}{]}\PY{o}{*}\PY{n}{x}\PY{p}{[}\PY{l+m+mi}{2}\PY{p}{]} \PY{o}{\PYZhy{}} \PY{n}{R}\PY{p}{[}\PY{l+m+mi}{1}\PY{p}{,}\PY{l+m+mi}{3}\PY{p}{]}\PY{o}{*}\PY{n}{x}\PY{p}{[}\PY{l+m+mi}{3}\PY{p}{]} \PY{p}{)}\PY{o}{/}\PY{n}{R}\PY{p}{[}\PY{l+m+mi}{1}\PY{p}{,}\PY{l+m+mi}{1}\PY{p}{]}
\PY{n}{x}\PY{p}{[}\PY{l+m+mi}{0}\PY{p}{]}  \PY{o}{=} \PY{p}{(}\PY{n}{b}\PY{p}{[}\PY{l+m+mi}{0}\PY{p}{]} \PY{o}{\PYZhy{}} \PY{n}{R}\PY{p}{[}\PY{l+m+mi}{0}\PY{p}{,}\PY{l+m+mi}{1}\PY{p}{]}\PY{o}{*}\PY{n}{x}\PY{p}{[}\PY{l+m+mi}{1}\PY{p}{]} \PY{o}{\PYZhy{}} \PY{n}{R}\PY{p}{[}\PY{l+m+mi}{0}\PY{p}{,}\PY{l+m+mi}{2}\PY{p}{]}\PY{o}{*}\PY{n}{x}\PY{p}{[}\PY{l+m+mi}{2}\PY{p}{]} \PY{o}{\PYZhy{}} \PY{n}{R}\PY{p}{[}\PY{l+m+mi}{0}\PY{p}{,}\PY{l+m+mi}{3}\PY{p}{]}\PY{o}{*}\PY{n}{x}\PY{p}{[}\PY{l+m+mi}{3}\PY{p}{]} \PY{p}{)}\PY{o}{/}\PY{n}{R}\PY{p}{[}\PY{l+m+mi}{0}\PY{p}{,}\PY{l+m+mi}{0}\PY{p}{]}

\PY{c+c1}{\PYZsh{} Visualize result}
\PY{n}{f}\PY{p}{,}\PY{n}{ax}\PY{o}{=}\PY{n}{plt}\PY{o}{.}\PY{n}{subplots}\PY{p}{(}\PY{l+m+mi}{1}\PY{p}{)}
\PY{n}{ax}\PY{o}{.}\PY{n}{plot}\PY{p}{(}\PY{n}{array}\PY{p}{(}\PY{n}{x0}\PY{p}{)}\PY{p}{,}\PY{n}{y}\PY{p}{,}\PY{l+s+s1}{\PYZsq{}}\PY{l+s+s1}{x}\PY{l+s+s1}{\PYZsq{}}\PY{p}{)}
\PY{n}{ax}\PY{o}{.}\PY{n}{plot}\PY{p}{(}\PY{n}{x0}\PY{p}{,}\PY{n}{A0}\PY{n+nd}{@x}\PY{p}{)}
\PY{n}{ax}\PY{o}{.}\PY{n}{xaxis}\PY{o}{.}\PY{n}{set\PYZus{}major\PYZus{}formatter}\PY{p}{(}\PY{n}{FuncFormatter}\PY{p}{(}
   \PY{k}{lambda} \PY{n}{val}\PY{p}{,}\PY{n}{pos}\PY{p}{:} \PY{l+s+s1}{\PYZsq{}}\PY{l+s+si}{\PYZob{}:.0g\PYZcb{}}\PY{l+s+s1}{\PYZdl{}}\PY{l+s+s1}{\PYZbs{}}\PY{l+s+s1}{pi\PYZdl{}}\PY{l+s+s1}{\PYZsq{}}\PY{o}{.}\PY{n}{format}\PY{p}{(}\PY{n}{val}\PY{o}{/}\PY{n}{np}\PY{o}{.}\PY{n}{pi}\PY{p}{)} \PY{k}{if} \PY{n}{val} \PY{o}{!=}\PY{l+m+mi}{0} \PY{k}{else} \PY{l+s+s1}{\PYZsq{}}\PY{l+s+s1}{0}\PY{l+s+s1}{\PYZsq{}}
\PY{p}{)}\PY{p}{)}
\PY{n}{ax}\PY{o}{.}\PY{n}{xaxis}\PY{o}{.}\PY{n}{set\PYZus{}major\PYZus{}locator}\PY{p}{(}\PY{n}{MultipleLocator}\PY{p}{(}\PY{n}{base}\PY{o}{=}\PY{n}{np}\PY{o}{.}\PY{n}{pi}\PY{p}{)}\PY{p}{)}

\PY{n}{plt}\PY{o}{.}\PY{n}{show}\PY{p}{(}\PY{p}{)}

\PY{n}{linalg}\PY{o}{.}\PY{n}{norm}\PY{p}{(}\PY{n}{A0}\PY{n+nd}{@x}\PY{o}{\PYZhy{}}\PY{n}{y}\PY{p}{,}\PY{l+m+mi}{2}\PY{p}{)}
\end{Verbatim}
\end{tcolorbox}

    \begin{Verbatim}[commandchars=\\\{\}]
(200, 4)
    \end{Verbatim}

    \begin{center}
    \adjustimage{max size={0.9\linewidth}{0.9\paperheight}}{output_6_1.png}
    \end{center}
    { \hspace*{\fill} \\}
    
            \begin{tcolorbox}[breakable, size=fbox, boxrule=.5pt, pad at break*=1mm, opacityfill=0]
\prompt{Out}{outcolor}{218}{\boxspacing}
\begin{Verbatim}[commandchars=\\\{\}]
1778.7782869889143
\end{Verbatim}
\end{tcolorbox}
        
    \begin{tcolorbox}[breakable, size=fbox, boxrule=1pt, pad at break*=1mm,colback=cellbackground, colframe=cellborder]
\prompt{In}{incolor}{220}{\boxspacing}
\begin{Verbatim}[commandchars=\\\{\}]
\PY{n}{U}\PY{p}{,}\PY{n}{S}\PY{p}{,}\PY{n}{V} \PY{o}{=} \PY{n}{linalg}\PY{o}{.}\PY{n}{svd}\PY{p}{(}\PY{n}{A}\PY{p}{,} \PY{n}{full\PYZus{}matrices}\PY{o}{=}\PY{k+kc}{True}\PY{p}{)}


\PY{c+c1}{\PYZsh{} Solve: S w = U b}
\PY{n}{b2} \PY{o}{=} \PY{n}{U}\PY{o}{.}\PY{n}{T}\PY{n+nd}{@b0}
\PY{n}{w}  \PY{o}{=} \PY{n}{zeros}\PY{p}{(}\PY{p}{[}\PY{l+m+mi}{4}\PY{p}{,}\PY{l+m+mi}{1}\PY{p}{]}\PY{p}{)}
\PY{n}{x} \PY{o}{=} \PY{n}{w}
\PY{k}{for} \PY{n}{i} \PY{o+ow}{in} \PY{n+nb}{range}\PY{p}{(}\PY{l+m+mi}{0}\PY{p}{,}\PY{l+m+mi}{4}\PY{p}{)}\PY{p}{:}
    \PY{n}{w}\PY{p}{[}\PY{n}{i}\PY{p}{]} \PY{o}{=} \PY{n}{b2}\PY{p}{[}\PY{n}{i}\PY{p}{]}\PY{o}{/}\PY{n}{S}\PY{p}{[}\PY{n}{i}\PY{p}{]}
\PY{n}{x} \PY{o}{=} \PY{n}{V}\PY{o}{.}\PY{n}{T}\PY{n+nd}{@w}


\PY{c+c1}{\PYZsh{} Visualize result}
\PY{n}{f}\PY{p}{,}\PY{n}{ax}\PY{o}{=}\PY{n}{plt}\PY{o}{.}\PY{n}{subplots}\PY{p}{(}\PY{l+m+mi}{1}\PY{p}{)}
\PY{n}{ax}\PY{o}{.}\PY{n}{plot}\PY{p}{(}\PY{n}{array}\PY{p}{(}\PY{n}{x0}\PY{p}{)}\PY{p}{,}\PY{n}{y}\PY{p}{,}\PY{l+s+s1}{\PYZsq{}}\PY{l+s+s1}{x}\PY{l+s+s1}{\PYZsq{}}\PY{p}{)}
\PY{n}{ax}\PY{o}{.}\PY{n}{plot}\PY{p}{(}\PY{n}{array}\PY{p}{(}\PY{n}{x0}\PY{p}{)}\PY{p}{,}\PY{n}{A0}\PY{n+nd}{@x}\PY{p}{)}
\PY{n}{ax}\PY{o}{.}\PY{n}{xaxis}\PY{o}{.}\PY{n}{set\PYZus{}major\PYZus{}formatter}\PY{p}{(}\PY{n}{FuncFormatter}\PY{p}{(}
   \PY{k}{lambda} \PY{n}{val}\PY{p}{,}\PY{n}{pos}\PY{p}{:} \PY{l+s+s1}{\PYZsq{}}\PY{l+s+si}{\PYZob{}:.0g\PYZcb{}}\PY{l+s+s1}{\PYZdl{}}\PY{l+s+s1}{\PYZbs{}}\PY{l+s+s1}{pi\PYZdl{}}\PY{l+s+s1}{\PYZsq{}}\PY{o}{.}\PY{n}{format}\PY{p}{(}\PY{n}{val}\PY{o}{/}\PY{n}{np}\PY{o}{.}\PY{n}{pi}\PY{p}{)} \PY{k}{if} \PY{n}{val} \PY{o}{!=}\PY{l+m+mi}{0} \PY{k}{else} \PY{l+s+s1}{\PYZsq{}}\PY{l+s+s1}{0}\PY{l+s+s1}{\PYZsq{}}
\PY{p}{)}\PY{p}{)}
\PY{n}{ax}\PY{o}{.}\PY{n}{xaxis}\PY{o}{.}\PY{n}{set\PYZus{}major\PYZus{}locator}\PY{p}{(}\PY{n}{MultipleLocator}\PY{p}{(}\PY{n}{base}\PY{o}{=}\PY{n}{np}\PY{o}{.}\PY{n}{pi}\PY{p}{)}\PY{p}{)}

\PY{n}{plt}\PY{o}{.}\PY{n}{show}\PY{p}{(}\PY{p}{)}

\PY{n}{linalg}\PY{o}{.}\PY{n}{norm}\PY{p}{(}\PY{n}{A0}\PY{n+nd}{@x}\PY{o}{\PYZhy{}}\PY{n}{y}\PY{p}{,}\PY{l+m+mi}{2}\PY{p}{)}
\end{Verbatim}
\end{tcolorbox}

    \begin{center}
    \adjustimage{max size={0.9\linewidth}{0.9\paperheight}}{output_7_0.png}
    \end{center}
    { \hspace*{\fill} \\}
    
            \begin{tcolorbox}[breakable, size=fbox, boxrule=.5pt, pad at break*=1mm, opacityfill=0]
\prompt{Out}{outcolor}{220}{\boxspacing}
\begin{Verbatim}[commandchars=\\\{\}]
1778.7782869889138
\end{Verbatim}
\end{tcolorbox}
        
    \section{Task 2}\label{task-2}

    Condition number for solving the equation \(Ax=b\). We consider
perturbations in the solution \(\delta x\) caused by perturbations in
the RHS \(\delta b\) i.e. \(A(x+\delta x)=b + \delta\). Since \(Ax=b\)
we should have \(A\delta x=\delta b\). Using SVD we get

\begin{align}
    U\Sigma V^T x &= b \\
    x &= V \Sigma^{-1} U^T b.
\end{align}

if \(b\) is a left singular vector \(b = u_j\) the RHS becomes
\(v_j \frac{1}{\sigma_j}\) so the norm becomes

\begin{align}
    || x  || &= || \frac{1}{\sigma_j} v_j || = \frac{1}{\sigma_j} ||v_j||.
\end{align}

Similarly if we use
\(\delta x = \delta b = \delta u_k = \varepsilon u_k\) we get

\begin{align}
    ||\delta x || = \frac{\varepsilon}{\sigma_k} ||v_k||.
\end{align}

Now if we choose \(b=u_1\) and \(\delta b = \varepsilon u_n\) and we
arrive at the qoutient

\begin{align}
\frac{|| \delta x ||}{|| x ||} = \varepsilon \kappa(A),
\end{align}

so we have found a vector pair \((b,\delta b)\) for which the bound

\begin{equation}
     \frac{|| \delta x ||}{|| x ||} \geq \kappa (A) \frac{|| \delta b ||}{|| b ||},
     \label{eq:bound}
\end{equation}

is tight.

    \section{Task 3}\label{task-3}

A "worst case" vector pair is when the bound in (\ref{eq:bound}) is
tight. TThus we compare \(x=A^{-1}b\) and
\(x + \delta x = A^{-1}(b+\delta b)\).

    \begin{tcolorbox}[breakable, size=fbox, boxrule=1pt, pad at break*=1mm,colback=cellbackground, colframe=cellborder]
\prompt{In}{incolor}{240}{\boxspacing}
\begin{Verbatim}[commandchars=\\\{\}]
\PY{k+kn}{from} \PY{n+nn}{scipy} \PY{k+kn}{import} \PY{n}{linalg} \PY{k}{as} \PY{n}{sl}
\PY{n}{H} \PY{o}{=} \PY{n}{sl}\PY{o}{.}\PY{n}{hilbert}\PY{p}{(}\PY{l+m+mi}{50}\PY{p}{)}
\PY{n}{U}\PY{p}{,}\PY{n}{S}\PY{p}{,}\PY{n}{V} \PY{o}{=} \PY{n}{linalg}\PY{o}{.}\PY{n}{svd}\PY{p}{(}\PY{n}{copy}\PY{p}{(}\PY{n}{H}\PY{p}{)}\PY{p}{)}
\PY{n}{b} \PY{o}{=} \PY{n}{U}\PY{p}{[}\PY{p}{:}\PY{p}{,}\PY{l+m+mi}{0}\PY{p}{]}
\PY{n}{db}\PY{o}{=} \PY{n}{b} \PY{o}{+} \PY{n}{U}\PY{p}{[}\PY{p}{:}\PY{p}{,}\PY{o}{\PYZhy{}}\PY{l+m+mi}{1}\PY{p}{]}
\PY{n}{x\PYZus{}0} \PY{o}{=} \PY{n}{sl}\PY{o}{.}\PY{n}{invhilbert}\PY{p}{(}\PY{l+m+mi}{50}\PY{p}{)}\PY{n+nd}{@b}
\PY{n}{x\PYZus{}1} \PY{o}{=} \PY{n}{sl}\PY{o}{.}\PY{n}{invhilbert}\PY{p}{(}\PY{l+m+mi}{50}\PY{p}{)}\PY{n+nd}{@db}
\PY{n}{linalg}\PY{o}{.}\PY{n}{norm}\PY{p}{(}\PY{n}{x\PYZus{}1}\PY{o}{\PYZhy{}}\PY{n}{x\PYZus{}0}\PY{p}{,}\PY{l+m+mi}{2}\PY{p}{)}
\end{Verbatim}
\end{tcolorbox}

            \begin{tcolorbox}[breakable, size=fbox, boxrule=.5pt, pad at break*=1mm, opacityfill=0]
\prompt{Out}{outcolor}{240}{\boxspacing}
\begin{Verbatim}[commandchars=\\\{\}]
1.3740438898204339e+73
\end{Verbatim}
\end{tcolorbox}
        
    \section{Task 4}\label{task-4}

Sylvesters criterion. A symmetric positive definite matrix (SPD) \(A\)
is defined

\begin{equation}
x^T A x > 0 \quad \forall x \neq \boldsymbol{0} 
\end{equation}

We consider the SPD matrix A of dimension \(nxn\)

\begin{equation}
A = \begin{bmatrix}
    a_11 & w^T \\
    w    & A_1
    \end{bmatrix},
\end{equation}

where \(a_11\) has dimension 1, w \(n-1\) and \(A_1\) is
\((n-1)x(n-1)\). To show that \(a_{11}\) is positive if \(A\) is
positive we consider the vector \(x_1=(\hat{x},\boldsymbol{0})\), where
\(\boldsymbol{0}\) is the zero vector of dimension \(n-1\) and
\(\hat{x} \neq 0\). Then, by the definition of an SPD matrix we have

\begin{equation}
x^T A x = \hat{x}^T A \hat{x} > 0.
\end{equation}

In the same way we may consider \(x_2=(0,\hat{x}_2)\) where
\(\hat{x}_2 \neq \boldsymbol{0}\) and again use the definited of an SPD
to show that \(A_1\) is SPD.

    \section{Task 5}\label{task-5}

A strictly diagonally dominant matrix \(A\) has the property

\begin{equation}
 a_{ii} = \sum_{j \neq i} a_{ij} \forall i.
 \label{eq:strict_def}
\end{equation}

To prove that A is nonsingular we consider

\begin{equation}
 a_{ij}u_j = 0,
 \label{eq:au}
\end{equation}

which should only have solutions for \(u_j=0\). (\ref{eq:au}) can be
rewritten by subtracting the diagonal of \(A\)

\begin{equation}
 a_{ii} u_i = - \sum_{j \neq i}a_{ij}u_j ,
\end{equation}

so

\begin{equation}
 a_{ii} = - \sum_{j \neq i}a_{ij} \frac{u_j}{u_i},
\end{equation}

for which we may consider the norm

\begin{equation}
 | a_{ii} | = | \sum_{j \neq i}a_{ij} \frac{u_j}{u_i} | \leq 
              | \sum_{j \neq i}a_{ij}| | \sum_{j \neq i} \frac{u_j}{u_i} |
 .
 \label{eq:end proof}
\end{equation}

Now if there exist some \(u_i \neq 0\) that has the largest absolute
value in \(u\) then (\ref{eq:end proof}) would contradict
\(\ref{eq:strict_def}\). On the other hand if all elements \(u_i\) in
\(u_j\) have the same magnitude this would also be a contradiction as
then \(| \sum_{j \neq i} \frac{u_j}{u_i} | < 1\).


    % Add a bibliography block to the postdoc
    
    
    
\end{document}
